\documentclass[11pt, oneside]{article}   	% use "amsart" instead of "article" for AMSLaTeX format
\usepackage[margin=1in]{geometry}                		% See geometry.pdf to learn the layout options. There are lots.
\geometry{a4paper}                   		% ... or a4paper or a5paper or ... 
%\geometry{landscape}                		% Activate for rotated page geometry
\usepackage[parfill]{parskip}    		% Activate to begin paragraphs with an empty line rather than an indent
\usepackage{graphicx}				% Use pdf, png, jpg, or eps§ with pdflatex; use eps in DVI mode
								% TeX will automatically convert eps --> pdf in pdflatex		
\usepackage{amssymb}
\usepackage{hyperref}          % hyperlink

%SetFonts

%SetFonts


\title{Thesis topic research}
\author{Lingjie}
%\date{}							% Activate to display a given date or no date

\begin{document}
\maketitle

\tableofcontents

\section{Overview}
    \href{https://www.springer.com/journal/10614}{Computational Economics}: 
    Research on applying Machine Learning techniques on economics issues 
    $\rightarrow$ focusing on the HOW (framework and usage)

    Proposed structure
    \begin{enumerate}
        \item Research on existing framework and algorithms
            \subitem[1.1] Explaining algorithms
            \subitem[1.2] Comparing performance
        \item Applying framework and algorithms to illustrate the benefit it brings to economic field
            \subitem[2.1] classic data (with known ground truth or baseline for comparison)
            \subitem[2.2] new dataset to illustrate usefulness
    \end{enumerate}

    Relevant keyword: computational economics, machine learning, economic + *topic keywords*

    Search words:
    \href{https://scholar.google.com.sg/scholar?q=
    machine+learning+on+economics&hl=en&as_sdt=0&as_vis=1&oi=scholart}{machine learning on economics}

    \subsection{Questions on writing thesis paper}

    How in-depth should an undergraduate thesis be? $\rightarrow$ what is the expectation?
    I saw some thesis simply compare off the shelve ML models without hyper-parameter tuning.

\section{Possible topics}

    \subsection{ML + Causal inference}

    \textbf{Motivation:} I think social sciences' unique selling point is in its causal inference study. The ability and
    focus on causal relationship instead of correlation sets a social scientist apart from computer scientist. \\
    Personally, I hope to further research on applying ML and other techniques to estimate heterogeneous causal effects. \\
    However, I have no experience in causal inference and I'm uncertain about how 'impactful' a causal paper might be,
    given that there is no ground truth on counter factual in the real world to compare with. \\
    Having said that, research on causal inference is of my top interest now.

    \subsubsection{Relevant papers}
    Framework
    \begin{itemize}
        \item \href{https://ideas.repec.org/p/ecl/stabus/3350.html}{Machine Learning for Estimating Heterogeneous Causal
            Effects}
        \item \href{https://imai.fas.harvard.edu/research/files/svm.pdf}{Estimating treatment effect heterogeneity in
            randomized program evaluation}
        \item \href{https://academic.oup.com/poq/article-abstract/76/3/491/1893905}{Modelling Heterogeneous Treatment
            Effects in Survey Experiments with Bayesian Additive Regression Trees}
        \item \href{https://oce-ovid-com.libproxy1.nus.edu.sg/article/00001648-202105000-00012/HTML}{Machine Learning
            for Causal Inference: On the Use of Cross-fit Estimators}
        \item \href{https://github.com/jvpoulos/causal-ml}{a list of research papers on causal ML}
    \end{itemize}

    Sample use case
    \begin{itemize}
        \item \href{https://academic-oup-com.libproxy1.nus.edu.sg/biostatistics/article/21/2/336/5631847}{Machine
            learning for causal inference in Biostatistics}
    \end{itemize}

    Open-source library
    \begin{itemize}
        \item \href{https://github.com/uber/causalml}{Uber causal ML}
        \item \href{https://github.com/microsoft/EconML}{Microsoft causal ML} and
            \href{https://www.microsoft.com/en-us/research/group/causal-inference/}{Microsoft Causality and Machine
            Learning}
        \item \href{https://www.altdeep.ai/p/causal-ml-minicourse}{paid course on causal ML}
    \end{itemize}

    \subsection{ML + Causal inference + Optimization}

    \textbf{Motivation:} This topic will be an extension of ML + Causal inference. In a business use case, usually the
    causal effect is not the final variable of interest. Business usually wants to maximise profit and/or minimise cost.
    Therefore, the causal effect is used to construct the optimization model which represents profit/cost function. \\
    However, currently business use correlation study to construct the optimization model instead of causation. It would
    be interesting to establish a framework from causal inference to optimization.

    \subsubsection{Relevant papers}

    Framework
    \begin{itemize}
        \item \href{https://www.oreilly.com/library/view/business-data-science/9781260452785/}{Business Data Science:
            Combining Machine Learning and Economics to Optimize, Automate, and Accelerate Business Decisions}
            $\rightarrow$ 'Understand how use ML tools in real world business problems, where causation matters more
            that correlation'
    \end{itemize}

    \subsection{ML + Forecasting}

    \textbf{Motivation:} This will be most classic use case for Machine Learning $\rightarrow$ prediction based problem.
    We can apply different ML techniques to improve the economic forecasting ability. \\
    Furthermore, it will be the easiest to write about given the current maturity in data science field 
    and most of my internship experience is prediction based. \\
    However, this topic is the least interesting to me as I have worked on various projects related to prediction. \\
    In addition, I think there will be little value add to the academic world given the massive research done on
    prediction problems.

    \subsubsection{Relevant papers}
    Sample use case
    \begin{itemize}
        \item \href{https://link-springer-com.libproxy1.nus.edu.sg/article/10.1007/s10614-021-10094-w}{Machine Learning
            in Economics and Finance}
    \end{itemize}

\section{Possible dataset}

    \subsection{Open data}

    Possible open sourced dataset (either through web scraping or direct download)

    \begin{itemize}
        \item Social media website (e.g. Twitter, Facebook, Linkedin)
        \item \href{https://www.google.com/covid19/mobility/}{Google mobility data}
        \item Financial data
            \subitem Bitcoin/stock market
            \subitem Financial news (NLP)
        \item Macro-economic indicators
            \subitem GDP/employment data
    \end{itemize}

    \subsection{Private data}

    \begin{enumerate}
        \item Singapore water consumption (domestic or non-domestic) $\rightarrow$ require further communication with
            IWP and PUB
            \subitem I had a previous internship with IWP and my researcher said we could discuss further for data
            access
    \end{enumerate}


\end{document}  
