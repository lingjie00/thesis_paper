\documentclass[11pt, oneside]{article}   	% use "amsart" instead of "article" for AMSLaTeX format
\usepackage[margin=1in]{geometry}                		% See geometry.pdf to learn the layout options. There are lots.
\geometry{a4paper}                   		% ... or a4paper or a5paper or ... 
%\geometry{landscape}                		% Activate for rotated page geometry
\usepackage[parfill]{parskip}    		% Activate to begin paragraphs with an empty line rather than an indent
\usepackage{graphicx}				% Use pdf, png, jpg, or eps§ with pdflatex; use eps in DVI mode
								% TeX will automatically convert eps --> pdf in pdflatex		
\usepackage{amssymb}
\usepackage{hyperref}          % hyperlink

%SetFonts

%SetFonts


\title{Thesis topic research}
\author{Lingjie}
%\date{}							% Activate to display a given date or no date

\begin{document}
\maketitle

\tableofcontents

\section{Overview}
    Applied research on solving economics issues with Machine Learning techniques.
    The thesis paper will focus on the framework and use case
    (HOW do we apply machine learning techniques and WHAT benefits we can gain).

    Proposed structure
    \begin{enumerate}
        \item Research on existing framework and algorithms
            \subitem[1.1] Explaining algorithms
            \subitem[1.2] Comparing performance
        \item Applying framework and algorithms to illustrate the benefit it brings to economic field
            \subitem[2.1] classic data (with known ground truth or baseline for comparison)
            \subitem[2.2] new dataset to illustrate usefulness
    \end{enumerate}

    Relevant keyword: \href{https://www.springer.com/journal/10614}{Computational Economics}, 
    machine learning, economic + *topic keywords*

    Search words:
    \href{https://scholar.google.com.sg/scholar?q=
    machine+learning+on+economics&hl=en&as_sdt=0&as_vis=1&oi=scholart}{machine learning on economics}

    \subsection{Some questions regarding thesis paper}

    How in-depth should an undergraduate thesis be? $\rightarrow$ what is the expectation?
    I saw some thesis simply compare off the shelve ML models without hyper-parameter tuning.

\section{Possible topics}

    \subsection{ML + Causal inference}

    \textbf{Motivation:}
    Social science is unique in its causal inference study. 
    Other data domains such as statistics and computer science focus on prediction instead causation.

    I hope to explore research topics on applying machine learning (and other techniques) to estimate
    heterogeneous causal effects.

    \textbf{Concerns:} I have no experience in causal inference and I'm uncertain how 'impactful' a
    causal paper needs to be.
    I have passed causal inference course in Coursera and read some discussions on causal inference.
    I will be taking EC4305 Applied Econometrics next semester to have a formal learning.

    \subsubsection{Relevant papers}
    Framework
    \begin{itemize}
        \item \href{https://ideas.repec.org/p/ecl/stabus/3350.html}{Machine Learning for Estimating Heterogeneous Causal
            Effects}
        \item \href{https://imai.fas.harvard.edu/research/files/svm.pdf}{Estimating treatment effect heterogeneity in
            randomized program evaluation}
        \item \href{https://academic.oup.com/poq/article-abstract/76/3/491/1893905}{Modelling Heterogeneous Treatment
            Effects in Survey Experiments with Bayesian Additive Regression Trees}
        \item \href{https://oce-ovid-com.libproxy1.nus.edu.sg/article/00001648-202105000-00012/HTML}{Machine Learning
            for Causal Inference: On the Use of Cross-fit Estimators}
        \item \href{https://github.com/jvpoulos/causal-ml}{a list of research papers on causal ML}
    \end{itemize}

    Sample use case
    \begin{itemize}
        \item \href{https://academic-oup-com.libproxy1.nus.edu.sg/biostatistics/article/21/2/336/5631847}{Machine
            learning for causal inference in Biostatistics}
    \end{itemize}

    Open-source library
    \begin{itemize}
        \item \href{https://github.com/uber/causalml}{Uber causal ML}
        \item \href{https://github.com/microsoft/EconML}{Microsoft causal ML} and
            \href{https://www.microsoft.com/en-us/research/group/causal-inference/}{Microsoft Causality and Machine
            Learning}
        \item \href{https://www.altdeep.ai/p/causal-ml-minicourse}{paid course on causal ML}
    \end{itemize}

    \subsection{ML + Forecasting}

    \textbf{Motivation:} Machine learning (and deep learning) introduces more toolkits for economists to apply on
    forecasting problems.

    Perhaps we can explore the benefit ML techniques brings to a specific domain or problem.

    \textbf{Concerns:} Although I'm most confident with forecasting models, I'm not sure if writing a thesis
    on this topic will be interesting. Furthermore, I'm not sure what is the value added to economic field
    with another model fitting.

    \subsubsection{Relevant papers}
    Sample use case
    \begin{itemize}
        \item \href{https://link-springer-com.libproxy1.nus.edu.sg/article/10.1007/s10614-021-10094-w}{Machine Learning
            in Economics and Finance}
    \end{itemize}

\section{Possible dataset}

    \subsection{Open data}

    Possible open sourced dataset (either through web scraping or direct download)

    \begin{itemize}
        \item Social media website (e.g. Twitter, Facebook, Linkedin)
        \item \href{https://www.google.com/covid19/mobility/}{Google mobility data}
        \item Financial data
            \subitem Bitcoin/stock market
            \subitem Financial news (NLP)
        \item Macro-economic indicators
            \subitem GDP/employment data
    \end{itemize}

\end{document}  
