\documentclass[11pt, oneside]{article}   	% use "amsart" instead of "article" for AMSLaTeX format
\usepackage[margin=1in]{geometry}                		% See geometry.pdf to learn the layout options. There are lots.
\geometry{a4paper}                   		% ... or a4paper or a5paper or ... 
%\geometry{landscape}                		% Activate for rotated page geometry
\usepackage[parfill]{parskip}    		% Activate to begin paragraphs with an empty line rather than an indent
\usepackage{graphicx}				% Use pdf, png, jpg, or eps§ with pdflatex; use eps in DVI mode
								% TeX will automatically convert eps --> pdf in pdflatex		
\usepackage{amsmath}
\usepackage{amssymb}
\usepackage{hyperref}          % hyperlink
\usepackage{color,soul}  % highlight

%SetFonts

%SetFonts


\title{Model}
\author{Lingjie}
%\date{}							% Activate to display a given date or no date

\begin{document}
\maketitle

\section{Stochastic discount factor (SDF)}

Arise from consumer maximising present and future utility by choosing asset amount

\begin{tabular}{l @{ := } l}
    $\xi$ & amount of asset \\
    $c_t$ & consumption at time $t$ \\
    $u(c_t)$ & utility at time $t$ \\
    $E_t(u)$ & expected utility at time $t$ \\
    $\beta$ & subjective discount factor \\
    $e_t$ & original consumption level at time $t$ \\
    $p_t$ & price of asset at time $t$ \\
    $x_t$ & payoff of asset at time $t$
\end{tabular}

\begin{align*}
    \max_{\xi} & ~u(c_t) + E_t\left[ \beta u(c_{t+1}) \right]~s.t.\\
    c_t &= e_t - p_t\xi \\
    c_{t+1} &= e_{t+1} + x_{t+1}\xi
\end{align*}

Solving for First Order Condition to find maxima

\begin{align*}
    p_tu'(c_t) &= E_t\left[ \beta u'(c_{t+1}) x_{t+1}  \right]\\
    \Rightarrow p_t &= E_t\left[ \beta \frac{u'(c_{t+1})}{u'(c_t)} x_{t+1}  \right] \\
    \Leftrightarrow p_t &= E_t \left[ m_{t+1} x_{t+1}  \right]
\end{align*}

where $m_{t+1}:=\beta\frac{u'(c_{t+1)}}{u'(c_t)}$ is defined as SDF

\section{No-Arbitrage Asset Pricing}

\subsection{No-arbitrage}

$m > 0 \Rightarrow$ No-arbitrage. Since marginal utility is assumed to be positive, $m > 0$.

No-arbitrage assumption $\Leftrightarrow$ there exist a $m_{t+1}$ such that

\begin{tabular}{l @{ := } l}
    $R^e_{t,i}$ & excess return of asset $i$ at time $t$ \\
    $R_{t,i}$ & actual return of asset $i$ at time $t$ \\
    $R^f_t$ & return of a risk free asset at time $t$
\end{tabular}

\begin{align*}
    R^e_{t+1, i} &= R_{t+1, i} - R^f_{t+1} = 0
\end{align*}

\subsection{exposure to systematic risk $\beta_{t, i}$ formulation}

Let $p_t = 0, x_{t+1} = R^e_{t+1, i} = 0 \Rightarrow E_t\left[ m_{t+1}R^e_{t+1, i}  \right] =0$

and
\begin{align*}
    E_t\left[ m_{t+1}R^e_{t+1, i}  \right] = 0 \Leftrightarrow E_t\left[ R^e_{t+1,i}  \right] 
    &= \left( - \frac{Cov_t(R^e_{t+1, i}, m_{t+1} )}{Var_t(m_{t+1})}  \right) \cdot \frac{Var_t(m_{t+1})}{E_t\left[ m_{t+1}  \right]} \\
    &= \beta_{t,i}\lambda_t
\end{align*}

to show $E_t\left[ R^e_{t+1, i}  \right] = \beta_{t, i}\lambda_t$

\begin{align*}
    E_t\left[ R^e_{t+1,i}  \right] &=
    \left( - \frac{Cov_t(R^e_{t+1, i}, m_{t+1} )}{Var_t(m_{t+1})}  \right) \cdot \frac{Var_t(m_{t+1})}{E_t\left[ m_{t+1}
    \right]} \\
                                   &= - \frac{Cov_t(R^e_{t+1, i}, m_{t+1})}{E_t\left[ m_{t+1}  \right]} \\
                                   &= - \frac{E_t\left[ R^e_{t+1, i}, m_{t+1}  \right] - E_t\left[ R^e_{t+1, i}
                                   \right]E_t\left[ m_{t+1}  \right]}{E_t\left[ m_{t+1}  \right]} \\
    \because E_t\left[ R^e_{t+1, i}, m_{t+1}  \right] &= 0 \\
    \Rightarrow E_t\left[ R^e_{t+1, i}  \right] &= E_t \left[ R^e_{t+1, i}  \right]~(shown)
\end{align*}

As shown, $E_t\left[ R^e_{t+1, i}  \right] = \beta_{t, i}\lambda_t$.

\begin{tabular}{l @{ := } l}
    $\beta_{t,i}$ & exposure to systematic risk \\
    $\lambda_t$ & price of risk \\
    $E_t[.]$  & expectation conditional on the information at time $t$
\end{tabular}

\subsection{SDF weight $\boldmath{\omega_t}$ and tangency portfolio $F_{t+1}$ formulation}

We will use

\begin{tabular}{l @{ := } l}
    $R^e_t$ & the vector representing $R^e_{t, i}$ for $i\in[1, N]$\\
    $\omega_t$ & the vector representing SDF weight $\omega_{t, i}$ for $i\in[1, N]$ \\
    $f_{t+1}$ & tangency portfolio, a scalar
\end{tabular}

Note $m_{t+1}$ is a scalar.

Now, since SDF is an affine transformation of the tangency portfolio ($m_{t+1} = a + b{f_{t+1}}$), we consider the
special case when $a=1, b=-1$, then

\begin{align*}
    m_{t+1} = 1 - f_{t+1} = 1 - \sum_{i=1}^N \omega_{t, i} R^e_{t+1, i} = 1 - {\omega_t}^T {R^e_{t+1}}
\end{align*}

where $f_{t+1} = \omega_t^TR^e_{t+1}$, the tangency portfolio is the weighted sum of excess returns.

\subsection{linking systematic risk $\beta_{t,i}$ to tangency portfolio $f_{t+1}$}

Now, since $m_{t+1} = 1 - f_{t+1}$, using previous result of 
\begin{align*}
    E_t[R^e_{t+1, i}] &= \left( - \frac{Cov_t(R^e_{t+1, i}, m_{t+1})}{Var_t(m_{t+1})}  \right) \cdot
    \frac{Var_t(m_{t+1})}{E_t[m_{t+1}]} \\
                      &= \left( - \frac{Cov_t(R^e_{t+1, i},(1-f_{t+1}))}{Var_t(1-f_{t+1})} \right) \cdot 
                      \frac{Var_t(1-f_{t+1})}{E_t[1-f_{t+1}]} \\
                      &= \frac{Cov_t(R^e_{t+1, i}, f_{t+1})}{Var_t(f_{t+1})} \cdot \frac{Var_t(f_{t+1})}{1 -
                      E_t[f_{t+1}]} \\
                      &= missing \\
                      &= \beta_{t, i} E_t[f_{t+1}]
\end{align*}

\hl{missing proving}

and $R^e_{t+1, i} = \beta_{t+1, i}\cdot f_{t+1} + \epsilon_{t+1, i}$

\section{Applying Generative adversarial network to estimate SDF}


Now, we basically are interested in finding two functions $\omega_t, \beta_{t, i}$ in order for us to find a $m_{t+1}$ such
that $E_t[m_{t+1}R^e_{t+1, i}]=0 \Rightarrow R^e_{t+1, i} = \beta_{t, i}f_{t+1}+\epsilon_{t+1, i}$ and achieve

\begin{enumerate}
    \item Explain cross-section of individual stock return $R^e_{t+1, i}$
    \item Construct conditional mean-variance efficient tagency portfolio $f_{t+1}(\omega_t)$
    \item Decompose stock returns into their predictable systematic componenet $\beta_{t, i}, f_{t+1}$ and their non-systematic unpredictable
        component $\epsilon_{t+1, i}$
\end{enumerate}

We will be using GAN (Generative adversarial network) methods to deduce $\omega_t$, which will construct $f_{t+1}$ and 
later will be used to deduce $\beta_{t, i}$.

\subsection{Generative adversarial Methods of Moments}

Now, in order for us to use GAN we will require a "Generator" and a "Discriminator". a generator will predict the
outcome we are interested in (in our case, the function $\omega_t$) while the discriminator will produce fake
data that meant to confuse the generator. The GAN model will continue to be trained till neither 
generator nor discriminator can improve their performance further.

\subsubsection{Generator: $\omega$}

\subsubsection{Discriminator: $g$}

We have identified our generator ($\omega_t$), now we will need to define our discriminator.

Previously we have justified that we can find a $m_{t+1}$ such that $E(m_{t+1}R^e_{t+1})=0$, 
which means that in a more general case we can always find a uncondtional condition where 

\begin{align*}
    E(m_{t+1}R^e_{t+1, i}g)=0
\end{align*}

\subsubsection{Model training and architecture}

\subsubsection{Input and Output data}

\end{document}
